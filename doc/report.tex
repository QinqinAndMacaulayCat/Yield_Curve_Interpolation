\documentclass{article}
\usepackage{amsmath}
\usepackage{booktabs}
\usepackage{tabularx}
\usepackage{geometry}
\geometry{a4paper, margin=1in}


\title{Yield Curve Interpolation Methods}
\author{Qinqin Huang}

\begin{document}

\maketitle

\section{Introduction}

\subsection{Motivation}

Yield curve construction is a fundamental task in finance, particularly in the valuation of fixed-income securities. A yield curve represents the relationship between interest rates (or yields) and different maturities. In practice, only a limited number of market instruments are available to derive the yield curve, necessitating the use of interpolation methods to estimate yields for maturities not directly observed in the market. 



\subsection{Objectives}


\section{Literature Review}



\subsection{Desirable Features of a Good Interpolated Yield Curve}


\begin{table}[h]
\centering
\caption{Desirable Features of a Good Interpolated Yield Curve}
\begin{tabularx}{\textwidth}{l c X}
\toprule
Feature & Description & Why Important \\
\midrule
No-arbitrage & Forward rates are non-negative & Ensures that the constructed yield curve does not imply arbitrage opportunities, which is crucial for accurate pricing of financial instruments. \\
Exactness & Reproduces market prices & Ensures that the yield curve accurately reflects the prices of the underlying market instruments, leading to reliable valuations. \\
Locality & Changes in one part of the curve do not affect distant parts & Allows for adjustments to specific segments of the yield curve without unintended consequences on other segments, enhancing flexibility in curve construction. \\
Stability & Small changes in input data lead to small changes in output & Ensures that the yield curve is robust to minor fluctuations in market data, providing confidence in the derived yields. \\
Local Hedging & Enables hedging of instruments using nearby maturities & Facilitates effective risk management by allowing traders to hedge positions using instruments with similar maturities, reducing basis risk. \\
\bottomrule
\end{tabularx}
\end{table}


\subsection{Common Interpolation Methods}


\section{Methodology}

\subsection{Data Collection}

\subsection{Implementation of Interpolation Methods}


\subsection{Evaluation Metrics}

\section{Results and Discussion}

\section{Conclusion}






\end{document}
